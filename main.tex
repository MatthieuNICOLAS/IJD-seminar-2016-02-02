% Copyright 2004 by Till Tantau <tantau@users.sourceforge.net>.
%
% In principle, this file can be redistributed and/or modified under
% the terms of the GNU Public License, version 2.
%
% However, this file is supposed to be a template to be modified
% for your own needs. For this reason, if you use this file as a
% template and not specifically distribute it as part of a another
% package/program, I grant the extra permission to freely copy and
% modify this file as you see fit and even to delete this copyright
% notice.

\documentclass{beamer}

\usepackage{tikz}
\usetikzlibrary{automata, matrix,shapes.multipart,shapes.geometric,fit,scopes,positioning}

\definecolor{blue-base}{RGB}{42, 79, 110}
\definecolor{purple-custom}{RGB}{54, 51, 119}
\definecolor{orange-custom}{RGB}{227, 164, 79}
\definecolor{yellow-custom}{RGB}{170, 142, 57}

% There are many different themes available for Beamer. A comprehensive
% list with examples is given here:
% http://deic.uab.es/~iblanes/beamer_gallery/index_by_theme.html
% You can uncomment the themes below if you would like to use a different
% one:
%\usetheme{AnnArbor}
%\usetheme{Antibes}
%\usetheme{Bergen}
%\usetheme{Berkeley}
%\usetheme{Berlin}
%\usetheme{Boadilla}
%\usetheme{boxes}
%\usetheme{CambridgeUS}
%\usetheme{Copenhagen}
%\usetheme{Darmstadt}
%\usetheme{default}
%\usetheme{Frankfurt}
%\usetheme{Goettingen}
%\usetheme{Hannover}
%\usetheme{Ilmenau}
%\usetheme{JuanLesPins}
%\usetheme{Luebeck}
\usetheme{Madrid}
%\usetheme{Malmoe}
%\usetheme{Marburg}
%\usetheme{Montpellier}
%\usetheme{PaloAlto}
%\usetheme{Pittsburgh}
%\usetheme{Rochester}
%\usetheme{Singapore}
%\usetheme{Szeged}
%\usetheme{Warsaw}

\title{ADT PLM}

% A subtitle is optional and this may be deleted
\subtitle{Programmer's Learning Machine}

\author{Matthieu~Nicolas}

\date{IJD Seminar, 2016-02-02}

\subject{Theoretical Computer Science}

\AtBeginSection[]
{
  \begin{frame}<beamer>{Outline}
    \tableofcontents[currentsection]
  \end{frame}
}

\begin{document}

\begin{frame}
  \titlepage
\end{frame}

\begin{frame}{Outline}
  \tableofcontents
  % You might wish to add the option [pausesections]
\end{frame}

\section{Presentation of PLM}

\subsection{Purposes}

\begin{frame}{Presentation of PLM}{Purposes}
  \begin{itemize}
  \item {
    Application to learn programming.
    \pause
  }
  \item {
    Allows students to progress at their own speed...
    \pause
  }
  \item {
    ... while the teacher helps the ones having trouble.
    \pause
  }
  \item {
    Used at TELECOM Nancy since 2008.
  }
  \end{itemize}
\end{frame}

\subsection{Demo}

% TODO: Record the video
\begin{frame}{Presentation of PLM}{Quick demo}
  \begin{center}
    \href{https://plm.telecomnancy.univ-lorraine.fr}{\includegraphics[scale=0.20]{img/screen-webPLM-1.png}}
  \end{center}
\end{frame}

\subsection{About PLM}

% QUESTION: Need slides for these infos or talk about them in the video?
\begin{frame}{Presentation of PLM}{12 lessons, 200 exercises}
  \begin{center}
    \includegraphics[scale=0.18]<1>{img/maze.png}
    \includegraphics[scale=0.18]<2>{img/hanoi.png}
    \includegraphics[scale=0.18]<3>{img/logo.png}
  \end{center}
\end{frame}

\begin{frame}{Presentation of PLM}{Languages and programming languages}
  \begin{itemize}
    \item {
      Available languages:
      \begin{itemize}
      \item English
      \item French
      \item Brazilian Portuguese
      \end{itemize}
    }
    \item[~]
    \item {
      Supported programming languages:
    }
  \end{itemize}
  \begin{center}
    \includegraphics[scale=0.16]{img/java.png}
    ~
    \includegraphics[scale=0.4]{img/scala.png}
    \includegraphics[scale=0.18]{img/python.png}
  \end{center}
\end{frame}

% TODO: Make this slide sexier
\begin{frame}{Presentation of PLM}{Evolution of the project}
  \begin{itemize}
    \item {
      Formerly a fat client
      \begin{itemize}
        \item { Written in Java }
      \end{itemize}
      \pause
    }
    \item[~]
    \item {
      Switch to a web application
      \begin{itemize}
        \item { Headless version of PLM }
        \item { Server implemented in Scala using \emph{PlayFramework} }
        \item { User interface written in Javascript using \emph{AngularJS} and \emph{Foundation} }
      \end{itemize}
    }
  \end{itemize}
  \begin{center}
    \includegraphics[scale=0.05]{img/play-logo.png}
    ~
    \includegraphics[scale=0.1]{img/foundation-angular.png}
  \end{center}
\end{frame}

\subsection{Architecture}

\begin{frame}{Presentation of PLM}{Application's architecture}
  \begin{center}
    \begin{tikzpicture}[
      scale=1,
      block/.style={
        rectangle, rounded corners=9pt,
        text width=8em,
        text centered,
        minimum height=3em,
        draw=black!50,
        fill=black!20
      },
      oauth/.style={
        block,
        color=white,
        fill=purple-custom
      }
    ]

      \node (user) { \includegraphics[scale=0.05]{img/user.pdf} };

      \node[block, right=1.5 of user] (webPLM) { webPLM };

      % OAuth-providers
      \node[oauth, scale=0.5, below=1.5 of user] (plm-accounts) { PLM-accounts };
      \node[scale=0.5, below=0.3 of plm-accounts] (google) { \includegraphics[scale=0.1]{img/google-logo.png} };
      \node[scale=0.5, below=0.3 of google] (github) { \includegraphics[scale=0.1]{img/github-logo.png} };
      \node[draw=purple-custom, densely dotted, fit=(plm-accounts) (google) (github)] (oauth-providers) {};
      \node[color=purple-custom, below=0.1 of oauth-providers] {OAuth-providers};

      \node[block, right=5 of oauth-providers] (plm-profiles) { PLM-profiles };

      \path[<->, shorten >=3pt, shorten <=3pt]
        (user) edge (webPLM)
        (user) edge node[left, scale=0.6] {authentificate} (oauth-providers)
        (webPLM) edge[bend right] node[above right, align=center, xshift=-5, scale=0.6] {retrieve user's \\ preferences} (plm-profiles.west)
        (webPLM) edge[bend left] node[below right, align=center, yshift=-15, xshift=-25, scale=0.6]  {validate \\ authentification} (oauth-providers.east);

    \end{tikzpicture}
  \end{center}
\end{frame}

\section{Assessment of user's code}

\subsection{Challenges}

\begin{frame}{Assessment of user's code}{Execution components}
  \begin{center}
    % QUESTION: how to picture it better?
    % Go further in details?
    \begin{tikzpicture}[
      scale=1,
      block/.style={
        rectangle, rounded corners=9pt,
        text width=8em,
        text centered,
        minimum height=3em,
        draw=black!50,
        fill=black!20
      }
    ]

      \node (user) { \includegraphics[scale=0.05]{img/user.pdf} };

      \node[block, right=2 of user] (plm-actor) { PLM-actor };
      \node[block, right=2 of plm-actor] (plm-engine) { PLM-engine };

      \node[draw, densely dotted, fit=(plm-actor) (plm-engine)] (webPLM) {};
      \node[below=0.1 of webPLM] {webPLM};

      \path[->, shorten >=2pt, shorten <=2pt]
        (user) edge node[sloped, anchor=center, above] {\tiny request execution} (plm-actor)
        (plm-actor) edge node[sloped, anchor=center, above] {\tiny start execution} (plm-engine)
        ([yshift=-0.2 cm]plm-engine.west) edge node[sloped, anchor=center, below] {\tiny send results} ([yshift=-0.2 cm]plm-actor.east)
        ([yshift=-0.2 cm]plm-actor.west) edge node[sloped, anchor=center, below] {\tiny relay results} ([yshift=-0.2 cm]user.east);
    \end{tikzpicture}
  \end{center}
\end{frame}

\begin{frame}{Assessment of user's code}{Limits}
  \begin{itemize}
  \item {
    Run on the same machine, same JVM
    \pause
  }
  \item[~]
  \item {
    How to protect ourselves from users' rookie mistakes?
    \begin{itemize}
    \item {
      Infinite loops
    }
    \end{itemize}
    \pause
  }
  \item {
    And from more malicious "mistakes"?
    \begin{itemize}
    \item {
      Infinite thread creation
    }
    \item {
      Endless file creation
    }
    \end{itemize}
    \pause
  }
  \item {
    And from \emph{System.exit(whatever)}?
    \pause
  }
  \item[~]
  \item {
    Scalability issues
  }
  \end{itemize}
\end{frame}

\subsection{Extraction of the execution component}

\begin{frame}{Assessment of user's code}{Chosen solution}
  \begin{itemize}
  \item {
    Delegate the execution to workers
    \begin{itemize}
    \item Called \emph{Judges} in the litterature
    \item Use headless version of PLM as well
    \item Execute user's code and send back result to webPLM
    \end{itemize}
    \pause
  }
  \item {
    \emph{Let it crash} strategy
    \begin{itemize}
    \item Prevent obvious issues with a security manager
    \item Handle timeout and crash
    \end{itemize}
    \pause
  }
  \item {
    Distribute workload using message queues
    \begin{itemize}
    \item One queue for requests
    \item One queue per result
    \end{itemize}
  }
  \end{itemize}
\end{frame}

\begin{frame}{Assessment of user's code}{Architecture with judges}
  \begin{center}
    \begin{tikzpicture}[
      scale=1,
      block/.style={
        rectangle, rounded corners=9pt,
        text width=8em,
        text centered,
        minimum height=3em,
        draw=black!50,
        fill=black!20
      },
      judge/.style={
        block,
        text width=4em
      }
    ]

      \node (user) { \includegraphics[scale=0.05]{img/user.pdf} };

      \node[block, right=3 of user] (plm-actor) { PLM-actor };
      \node[block, right=1 of plm-actor] (plm-engine) { PLM-engine };

      \node[draw, densely dotted, fit=(plm-actor) (plm-engine)] {};
      \node[fit=(plm-actor) (plm-engine), yshift=-30, xshift=30] {webPLM};

      \node[below=1.5 of plm-actor] (center-mq) {};

      \node[right=0.05 of center-mq] (requests-mq) { \includegraphics[angle=-90, scale=0.3]{img/message-queue-2.png} };
      \node[right=0.05 of center-mq, xshift=20] {\tiny Requests MQ};

      \node[left=0.05 of center-mq] (reply-mq) { \includegraphics[angle=90, scale=0.3]{img/message-queue-2.png} };
      \node[left=0.05 of center-mq, xshift=-20] {\tiny Reply MQ};

      \node[judge, below=1.5 of center-mq] (judge1) {Judge 1};
      \node[judge, right=of judge1] (judge2) {Judge 2};
      \node[right=of judge2] (judge3) {...};

      \node[draw, densely dotted, fit=(judge1) (judge2) (judge3)] {};
      \node[fit=(judge1) (judge2) (judge3), yshift=-35] {Judges};

      \path[->, shorten >=2pt, shorten <=2pt]
        (user) edge node[sloped, anchor=center, above] {\tiny request execution} (plm-actor)
        (plm-actor) edge node[sloped, anchor=center, above] {\tiny save result} (plm-engine)
        ([yshift=-0.2 cm]plm-actor.west) edge node[sloped, anchor=center, below] {\tiny relay results} ([yshift=-0.2 cm]user.east);
      \path[->, shorten >=2pt, shorten <=2pt]
        (plm-actor.south) edge[bend left] node[below right] {\tiny submit request} (requests-mq.north)
        (requests-mq.south) edge[bend left] node[right] {\tiny retrieve request} (judge1.north)
        (judge1.north) edge[bend left] node[left] {\tiny send result} (reply-mq.south)
        (reply-mq.north) edge[bend left] node[left] {\tiny retrieve result} (plm-actor.south);
    \end{tikzpicture}
  \end{center}
\end{frame}

% TODO: Find a subtitle
\begin{frame}{Assessment of user's code}{Pros and cons}
  \begin{itemize}
  \item {
    Pros:
    \begin{itemize}
    \item Allow to run code without impacting webPLM's performances
    \item Meet the scalability requirements
    \end{itemize}
    \pause
  }
  \item {
    Cons:
    \begin{itemize}
    \item Make sure to use the right version of PLM
    \item Need to deploy them easily
    \item Should be able to reset them
    \item Have to restrict their resources usage
    \end{itemize}
  }
  \end{itemize}
\end{frame}

\subsection{Docker}

\begin{frame}{Assessment of user's code}{Docker}
  \begin{itemize}
  \item {
    Lightweight virtualization tool
  }
  \item {
    Build image of your application
  }
  \item {
    Run containers based on images
  }
  \end{itemize}
  \begin{center}
    \includegraphics[scale=0.2]{img/docker-logo.png}
  \end{center}
\end{frame}

\begin{frame}{Assessment of user's code}{Example of Dockerfile}
  \begin{itemize}
  \item {
    Dockerfiles describe how to set up the application
  }
  \begin{center}
    \includegraphics[scale=0.12]{img/dockerfile.png}
  \end{center}
  \item {
    Run \emph{docker build \textcolor{red}{-t tag} \textcolor{blue}{/path/to/Dockerfile}} to build  the image
  }
  \item {
    Start containers with \emph{docker run \textcolor{red}{tag}}
  }
  \end{itemize}
\end{frame}

\begin{frame}{Assessment of user's code}{More about \emph{docker run}}
  \begin{itemize}
  \item {
    Can also manage
    \begin{itemize}
    \item {
      Ports
      \pause
    }
    \item {
      Volumes
      \pause
    }
    \item {
      Links between containers
      \pause
    }
    \item {
      Environment variables
    }
    \item {
      Runtime constraints on resources
    }
    \item {
      Restart policies
    }
    \item {
      And a \textbf{lot more}
      \pause
    }
    \end{itemize}
  }
  \item[~]
  \item {
    Commands can become quite complex
  }
  \end{itemize}
  \begin{center} {
    \emph{docker run \textcolor{blue}{-p 443:9443} \textcolor{red}{-link plm-accounts:accounts} \textcolor{green}{-v \raise.17ex\hbox{$\scriptstyle\sim$}/webPLM/logs/:/app/webplm-dist/logs} webPLM}
  }
  \end{center}
\end{frame}

\begin{frame}{Assessment of user's code}{Docker-compose}
  \begin{itemize}
  \item {
    Tool to easily deploy multi-containers applications
  }
  \begin{center}
    \includegraphics[scale=0.14]{img/docker-compose-2.png}
  \end{center}
  \item {
    Deploy environment with \emph{docker-compose up}
  }
  \end{itemize}
\end{frame}

\begin{frame}{Assessment of user's code}{Docker in our case}
  \begin{itemize}
  \item{
    Deploy easily all components
  }
  \item {
    Restart judges automatically
  }
  \item {
    Limit users' mischiefs
  }
  \end{itemize}
\end{frame}

\section{Result}

\begin{frame}{Result}{Current architecture}
  \begin{center}
    \begin{tikzpicture}[
      scale=1,
      block/.style={
        rectangle, rounded corners=9pt,
        text width=6em,
        text centered,
        minimum height=3em,
        draw=black!50,
        fill=black!20
      },
      oauth/.style={
        block,
        color=white,
        fill=purple-custom
      },
      judge/.style={
        block,
        text width=4em
      }
    ]

      \node (user) { \includegraphics[scale=0.05]{img/user.pdf} };
      \node[block, right=of user] (nginx) { nginx };

      %%%%%%%%%%%%%%%%%%
      % webPLMs
      %%%%%%%%%%%%%%%%%%
      %\node[block, above right=of nginx, yshift=-30] (webPLM1) { webPLM-1 };
      %\node[block, below right=of nginx, yshift=30] (webPLM2) { webPLM-2 };
      %\node[draw, densely dotted, fit=(webPLM1) (webPLM2)] {};
      %\node[fit=(webPLM1) (webPLM2)] (webPLMs) {webPLMs};
      \node[block, right=of nginx] (webPLM) {webPLM};

      \path[<->, shorten >=3pt, shorten <=3pt]
        (user) edge (nginx)
        (nginx) edge (webPLM);

      \node[block, right=of webPLM] (plm-profiles) { PLM-profiles };

      \path[<->, shorten >=3pt, shorten <=3pt]
        (webPLM.east) edge (plm-profiles.west);

      %%%%%%%%%%%%%%%%%%
      % OAuth-providers
      %%%%%%%%%%%%%%%%%%
      \node[oauth, scale=0.5, below=of nginx] (plm-accounts) { PLM-accounts };
      \node[scale=0.5, below=0.3 of plm-accounts] (google) { \includegraphics[scale=0.1]{img/google-logo.png} };
      \node[scale=0.5, below=0.3 of google] (github) { \includegraphics[scale=0.1]{img/github-logo.png} };
      \node[draw=purple-custom, densely dotted, fit=(plm-accounts) (google) (github)] (oauth-providers) {};
      \node[color=purple-custom, below=0.1 of oauth-providers] {OAuth-providers};

      \path[<->, shorten >=3pt, shorten <=3pt]
        (user) edge[bend right] (oauth-providers)
        (webPLM.south)[xshift=-10] edge[bend left] (oauth-providers.east);

      %%%%%%%%%%%%%%%%%%
      % Judges
      %%%%%%%%%%%%%%%%%%
      \node[below right =0.8 of webPLM.south, xshift=10, yshift=-35] (mq) {\includegraphics[scale=0.3]{img/message-queue-2.png}};
      \node[block, below right=0.5 of mq] (judges) {Judges};

      \path[<->, shorten >=2pt, shorten <=2pt]
        (webPLM.south) edge[bend right] (mq.west)
        (mq) edge[bend left] (judges);
    \end{tikzpicture}
  \end{center}
\end{frame}

\begin{frame}{Result}{Live-session in TELECOM Nancy}
  \begin{itemize}
  \item {
    Used in TELECOM Nancy in September 2015
  }
  \item {
    30 hours of live testing with 100 students.
    \pause
  }
  \item[~]
  \item {
    Engine is (almost) working fine...
  }
  \item {
    ... but user experience needs to be improved!
  }
  \end{itemize}
\end{frame}

% TODO: think of a transition between these 2 slides

\begin{frame}{Result}{Live-session in TELECOM Nancy}
  \begin{itemize}
  \item {
    Can't cope with the workload.
    \pause
  }
  \item {
    No tools for monitoring set up...
    \pause
  }
  \item {
    ... so the bottleneck is unknown.
  }
  \end{itemize}
\end{frame}

\section{Next steps}

\begin{frame}{Next steps}{Refactor the code}
  \begin{itemize}
  \item {
    Rushed to release a stable version before September 2015...
  }
  \item {
    Needed to clean some parts of the code.
  }
  \item {
    Standardized behavior of local and server mode.
  }
  \end{itemize}
\end{frame}

\begin{frame}{Next steps}{Simplify workflow to adapt the content}
  \begin{itemize}
  \item {
    Store most of content inside PLM.
  }
  \item {
    Heavy and error prone workflow.
  }
  \item {
    Need to extract the content from PLM's jar.
  }
  \item {
    Allow to implement an exercise editor.
  }
  \end{itemize}
\end{frame}

\begin{frame}{Next steps}{Solve performance issues}
  \begin{itemize}
  \item {
    Set up some monitoring tools.
  }
  \item {
    Perform some load testing to identify the bottleneck.
  }
  \end{itemize}
\end{frame}

\section*{Thanks}

\begin{frame}{Questions}
  \begin{center}
    Thanks for your attention, any questions?
  \end{center}
\end{frame}

\end{document}
